\documentclass[a4paper]{jsarticle}
%
\usepackage[dvipdfmx]{graphicx}
\usepackage{amsmath,amssymb}
\usepackage{bm}
\usepackage{ascmac}
%
\setlength{\textwidth}{\fullwidth}
\setlength{\textheight}{40\baselineskip}
\addtolength{\textheight}{\topskip}
\setlength{\voffset}{-0.2in}
\setlength{\topmargin}{0pt}
\setlength{\headheight}{0pt}
\setlength{\headsep}{0pt}
%
\newcommand{\divergence}{\mathrm{div}\,}  %ダイバージェンス
\newcommand{\grad}{\mathrm{grad}\,}  %グラディエント
\newcommand{\rot}{\mathrm{rot}\,}  %ローテーション
%
\title{情報工学基礎 CSCW}
\author{1955008   	佐原優衣}
\date{}

\begin{document}
\maketitle

\section*{CSCWとは}
CSCWとはComputer Supported Cooperative Work(コンピューター援用共同作業)のことの省略形であり,
\begin{itemize}
  \item item1
  \item item2
  \item ...
  \item itemN
\end{itemize}


%\begin{figure}
  %\centering
  %\includegraphics[width=5cm]{apple.jpg}
  %\caption{林檎の図}
  %\label{app}
%\end{figure}

\section{おわりに}
文献\cite{key1}

\begin{thebibliography}{3}
  \bibitem{key1} {吉野考, 宗森純ら, 電気情報通信学会「知識ベース」, 電気情報通信学会, 2010}
  \bibitem{key2} 参考文献の名前・著者2
  \bibitem{keyN} 参考文献の名前・著者N
\end{thebibliography}

\end{document}