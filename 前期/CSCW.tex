\documentclass[a4paper]{jsarticle}
%
\usepackage[dvipdfmx]{graphicx}
\usepackage{amsmath,amssymb}
\usepackage{bm}
\usepackage{ascmac}
%
\setlength{\textwidth}{\fullwidth}
\setlength{\textheight}{40\baselineskip}
\addtolength{\textheight}{\topskip}
\setlength{\voffset}{-0.2in}
\setlength{\topmargin}{0pt}
\setlength{\headheight}{0pt}
\setlength{\headsep}{0pt}
%
\newcommand{\divergence}{\mathrm{div}\,}  %ダイバージェンス
\newcommand{\grad}{\mathrm{grad}\,}  %グラディエント
\newcommand{\rot}{\mathrm{rot}\,}  %ローテーション
%
\title{情報工学基礎 CSCW}
\author{1955008   	佐原優衣}
\date{}

\begin{document}
\maketitle

\section*{CSCWとは}
CSCWとはComputer Supported Cooperative Workの省略形であり,協調作業のコンピュータによる支援の意味をもつ。前半がグループウェアの研究分野で,後半が協調作業の研究分野で,二つの研究分野が合わさった形である。

CWはグループウェアを含む情報通信技術を利用した様々な影響の調査分析を研究することである。
例として次に示す研究が行われている。
\begin{itemize}
  \item グループウェアが社会に及ぼす影響の調査分析
  \item グループウェアの効果や影響を研究する手法の開発
  \item Webやメールに蓄積された膨大なデータを利用した生産性の向上や業務改善手法の提案
\end{itemize}


グループウェアを用いて遠隔情報を共有したり,電子会議を行ったりするなど,たくさんの技術で利便性が上昇したにもかかわらず,実社会では効率的な運用が行われていないことの方が多い。効率的な運営はどうしたら良いのかの研究もまた行われている。これまでの研究から,インフォーマルコミュニケーションやアウェアネスの不足が問題であることがわかってきている。

%\begin{figure}
  %\centering
  %\includegraphics[width=5cm]{apple.jpg}
  %\caption{林檎の図}
  %\label{app}
%\end{figure}
アウェアネスの一手段として,人の有無に反応する置物が開発された。その時に一度設置はしたが,後になって撤去してほしいということになった。その原因は監視されているような気分になるというプライバシーの問題であった。今後,このプライバシーとの兼ね合いもcscwでは考えていかなければならない問題である。




\end{document}